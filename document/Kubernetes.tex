\documentclass[11pt,a4paper]{article}
\usepackage[utf8]{inputenc}
\usepackage[T1]{fontenc}
\usepackage{lmodern}
\usepackage[ngerman]{babel}
\usepackage{babelbib}
\usepackage{hologo}
\usepackage{csquotes}
\usepackage{amsmath}
\usepackage{amssymb}
\usepackage{amsthm}

\begin{document}

\section{Deckblatt}
\section{Inhaltsverzeichnis}



\section{Zusammenfassung}

\section{Einleitung}

\section{Hauptteil}

\section{Fazit}

\section{Ausblick}

\section{Anhang}


\section{Formatbeispiele}
\subsection{Eine Aufzählung}

\begin{enumerate}
	\item die \textbf{original PDF Vorlage} (also diese Datei),
	\item das von Ihnen erstellte \textbf{\hologo{LaTeX} Dokument} (also die .tex Datei),
	\item die von Ihnen erstellte \textbf{Literatur Datenbank} (also die .bib Datei), und
	\item das von Ihnen \textbf{kompilierte Dokument} (also die erzeugte .pdf Datei).
\end{enumerate}


\subsection{Tabellen}
Tabelle \ref{HdRTab} gibt eine kleine Übersicht über verschiedene Charaktere und Gruppierungen in \emph{Der Herr der Ringe}.

\begin{table}[b]
	\begin{tabular}{|r|c|c|c|c|}
		\hline
				& Gollum	& Legolas	& Sauron	& Gandalf \\
		\hline
		\hline
		Hobbit	& ja	& nein	& nein	& nein \\
		\hline
		Ringträger	& am Finger	& nein	& am Finger	& in der Hand \\
		\hline
		Gemeinschaft des Ringes	& nein	& ja	& nein	& ja \\
		\hline
	\end{tabular}
	\caption{Verschiedene Charaktere und Gruppierungen in \emph{Der Herr der Ringe}.}
	\label{HdRTab}
\end{table}

\subsection{Formeln}
Die bekannten Fibonacci-Zahlen $F_n$ sind wie folgt definiert.
\newtheorem{defn}{Definition}
\begin{defn}\label{fibonacci}
	Es sei $F_1 = F_2 = 1$ und $F_n = F_{n-1} + F_{n-2}$ für $n \geq 3$.
\end{defn}
Fibonacci-Zahlen treten sowohl in der Natur als auch in vielen theoretischen Anwendungen auf. Die folgende Formel für die in Definition \ref{fibonacci} wurde von verschiedenen Mathematikern im 18. und 19. Jahrhundert entdeckt. Es sei $\phi = \frac{1 + \sqrt{5}}{2}$ (der goldene Schnitt) und $\psi = \frac{1 - \sqrt{5}}{2}$. Dann gilt
\[ \forall n \in \mathbb{N}: \quad F_n = \frac{\phi^n - \psi^n}{\sqrt{5}}. \]

\subsection{Zitieren}
In Ihrer Abschlussarbeit werden Sie existierende Literatur bearbeiten und zitieren müssen. Das korrekte Formatieren übernimmt \hologo{LaTeX} (beziehungsweise \hologo{BibTeX}) für Sie. Dabei werden Sie vermutlich nicht aus dem Buch von Tolki-\linebreak en \cite{HdR1954} zitieren, aber möglicherweise aus einem Buch zum Übersetzerbau \cite{compilers2006}.

\bibliographystyle{babplain-fl}
\bibliography{literature}

\end{document}