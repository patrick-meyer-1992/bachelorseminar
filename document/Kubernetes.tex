% TODO:
% Referenzen für Kapitel und Abbildungen kontrollieren und überarbeiten
% Beschriftungen für Abbildungen ergänzen

\documentclass[11pt,a4paper]{article}
\usepackage[utf8]{inputenc}
\usepackage[T1]{fontenc}
\usepackage{lmodern}
\usepackage[ngerman]{babel}
\usepackage{graphicx}
% \usepackage{babelbib}
\usepackage{hologo}
\usepackage{csquotes}
\usepackage{amsmath}
\usepackage{amssymb}
\usepackage{amsthm}
\usepackage{biblatex}
\addbibresource{literature.bib}

\begin{document}

\section{Deckblatt}
\section{Inhaltsverzeichnis}



\section{Zusammenfassung}

\section{Einleitung}
\label{sec:einleitung}
Die vorliegende Arbeit soll einen Überblick über das Container-Orchestrierungs-Tool \emph{Kubernetes} geben. 
Dazu werden zunächst Historie und grundlegende Funktionsprinzipien von Kubernetes beschrieben, die anschließend
anhand einer beispielhaften Anwendung aus dem Bereich verteilter, agentenbasierter Simulationen demonstriert werden.

Zunächst soll die Historie betrachtet werden, vor deren Hintergrund Kubernetes entstanden ist.
Das stetige Ziel, das in sich durch diese Historie zieht, ist das Bereitstellen von Applikationen.
Kominos et al. \cite{7899247} beschreiben die 1960er Jahre als Ursprung des Cloud Computing.
Zu dieser Zeit liefen Applikationen oft auf einer gemeinsamen Hardware und die Ressourcenverteilung erfolgte unmittelbar durch das zugrundeliegende Betriebssystem.
Ein derartiges Szenario (auch als ``Bare-metal'' bezeichnet) bringt in der Regel mehrere Herausforderungen mit sich.
Zum einen kann es schnell zu Abhängigkeitskonflikten zwischen den verschiedenen Applikationen geben.
Ist eine Applikation darauf angewiesen eine bestimmte Abhängigkeit in der Version <1.4 zu nutzen und eine andere
benötigt eine Version \(\geq 1.4\) derselben Abhängigkeit, lässt sich das nur schwer vereinbaren.
Zum anderen erschwert dieser Aufbau eine effiziente Ressourcenverteilung. Selbst wenn eine Applikation alleine auf einem Rechner betrieben wird (z.B. um Konflikte zu anderen
Applikationen zu vermeiden), muss dieser Rechner ausreichende Kapazitäten haben, um die höchsten Anfragespitzen an diese Applikation zu verarbeiten.
Selbst dann wenn diese Anfragespitzen nur äußerst selten erreicht werden.

Eine Weiterentwicklung dieser Bare-metal-Architektur waren virtuelle Maschinen. Virtuelle Maschinen erlauben es, auf derselben Hardware mehrere virtuelle Instanzen eines
Systems zu nutzen. Die Instanzen nutzen dabei nicht direkt die physischen Ressourcen, sondern lediglich virtuelle, die durch Software nachgebildet werden \cite{kofler2021docker}.
Dadurch, dass verschiedene Applikationen in verschiedenen virtuellen Maschinen betrieben werden, können Abhängigkeitskonflikte zwischen Applikationen
vermieden werden. Weiterhin sind virtuelle Maschinen leichter auszubringen und zu administrieren als Bare-metal-Lösungen.
Grundsätzlich können mit diesem Ansatz Applikationen auch flexibler und skalierbarer bereitgestellt werden.
Leistungsfähige Hardware kann Anfragespitzen einer Applikation in einer virtuellen Maschine abfangen und zu anderen Zeiten anderen virtuellen Maschinen mit anderen
Applikationen diese Ressourcen zuweisen.
Obwohl das schon einige Verbesserungen im Vergleich zu Bare-metal-Lösungen sind, haben auch virtuelle Maschinen ihre eigenen Limitationen.
Da virtuelle Maschinen jeweils Ressourcen für ein eigenes Betriebssystem beanspruchen, geht vergleichsweise viel Leistung verloren, 
die besser in der Applikation genutzt werden könnte.

Die nächste Stufe der Weiterentwicklung waren (Software-)Container. Container haben einige Gemeinsamkeiten mit virtuellen Maschinen, doch der wesentliche Unterschied ist,
dass Container deutlich leichtgewichtiger sind. Das wird unter anderem dadurch erreicht, dass sie statt ein komplett eigenes Betriebssystem zu verwenden, 
Anteile des Host-Betriebssystems mitnutzen. Dadurch sind die ``Baupläne'' (sog. Images) für Container im Allgemeinen kleiner als für klassische virtuelle Maschinen.
Durch den verringerten Overhead können auf gleicher Hardware auch mehr Container parallel betrieben werden, als es mit virtuellen Maschinen möglich wäre.
Durch diesen Umstand und dadurch, dass Container meist innerhalb weniger Sekunden gestartet werden können, sind diese besonders geeignet,
um flexibel auf unterschiedliche Anfrageintensitäten zu reagieren und Ressourcen effizient zu verteilen \cite{kofler2021docker}.

Einzelne Container führen in der Regel nur eine klar umgrenzte Aufgabe aus (z.B. Betrieb einer Datenbank oder eines Webservers). Da sie in sich geschlossen sind,
können sie auch unabhängig von anderen Containern repliziert werden. Um eine vollständige Anwendung zu erhalten, betreibt man daher oft mehrere Container,
die im Verbund zusammenwirken; eine sogenannte Microservice-Architektur. Um den hohen Anforderungen an Rechenleistung moderner Anwendungen gerecht zu werden,
werden die Container dabei oft auf viele Rechner verteilt (sog. horizontale Skalierung). % Horizontale und vertikale Skalierung noch erklären?
Die Verwaltung einer solchen verteilten Microservice-Architektur kann schnell komplex werden, weshalb verschiedene Werkzeuge zur Unterstützung entwickelt wurden.

Eines der heute populärsten Werkzeuge dieser Art ist Kubernetes. Im Folgenden soll die Funktionsweise von Kubernetes beschrieben und schließlich anhand eines Beispiels 
verdeutlicht werden.

\section{Historie von Kubernetes}
Kubernetes ist eine Software, die sich der Ausbringung (Deployment), der Skalierung und der Steuerung (Management) von Containern widmet.
Dieser Vorgang wird auch als Container-Orchestrierung oder auch Container-Management bezeichnet \cite{Bisong2019}.
In diesem Zusammehang wird Kubernetes auch als \emph{Betriebssystem der Cloud} bezeichnet \cite{Schmeling_Dargatz_2022}. 
Entwickelt wurde Kubernetes von Google. Dort wurde seit den frühen 2000er Jahren ein selbst entwickeltes Container-Management-System namens \emph{Borg} genutzt,
um Googles Services möglichst performant zu betreiben. 
Im Jahr 2013 wurd \emph{Borg} von dessen Nachfolger \emph{Omega} abgelöst. Im selben Jahr erschien auch Docker, was ein wesentlicher Baustein des Kubernetes 
Projekts werden sollte.

Die drei bei Google beschäftigten Ingenieure Craig McLuckie, Joe Beda und Brendan Burns setzten sich damals dafür ein,
die bei \emph{Borg} und \emph{Omega} gemachten Erfahrungen zu nutzen, um ein leichter bedienbares Container-Orchestrierungs-Tool 
mit nutzerfreundlicher Schnittstelle zu entwickeln.

2014 wurde Kubernetes als eine Open-Source-Variante von \emph{Borg} veröffentlicht und schließlich 2015 mit der Veröffentlichung von Kubernetes 1.0
von Google an die Cloud Native Computing Foundation (CNCF, https://www.cncf.io/) gespendet. In den folgenden Jahren wuchs der Nutzerkreis von Kubernetes
rasch an und es konnte sich deutlich gegen konkurrierende Anwendungen durchsetzen.

Heutzutage ist Kubernetes, nach Linux, das zweitgrößte Open-Source Projekt der Welt und wird von 71\% der Fortune 100 Unternehmen genutzt. % (https://www.cncf.io/reports/kubernetes-project-journey-report/)
% https://www.ibm.com/think/topics/kubernetes-history abgerufen am 28.11.2024

\section{Container}
\label{sec:Container}

Um den Begriff des (Docker-)Containers zu verstehen, wird zunächst der Begriff \emph{Image} benötigt.
Ein Image kann wie in der Einleitung bereits beschrieben als eine Art Bauplan für einen Container gesehen werden.
Es stellt ein read-only Dateisystem als Basis für den Container zur Verfügung. Der Container nimmt an seinem Image
keine Änderungen vor. Stattdessen wird jedes Hinzufügen und jede Änderung während der Laufzeit des Containers in einem
getrennten Overlay-System behandelt. Dadurch können beliebig viele Container von demselben Image abgeleitet werden \cite{kofler2021docker}. 

Für viele populäre Anwendungen (z.B. nginx, ubuntu oder nodejs) gibt es vorgefertigte Images, die kostenfrei vom DockerHub % Quelle ergänzen
heruntergeladen werden können.
Für einen konkreten Anwendungsfall kann es sinnvoll sein, ein bestehendes Basis-Image für die eigenen Zwecke anzupassen.
Dies ist mit Hilfe eines sogenannten \emph{Dockerfile}s möglich. Darin wird definiert, wie ein Base-Image z.B. durch das Hinzufügen von Dateien,
oder dem Ändern von Umgebungsvariablen angepasst werden soll. Dieses \emph{Dockerfile} wird dann für den \emph{build}
eines neuen Images verwendet.
% Bild von Workflow einbinden?

Aus dem erstellten Image können dann Container abgeleitet werden. Ein Container ist insofern ``flüchtig'' als dass er nach seiner
Beendigung restlos gelöscht wird. Bei einem Neustart enthält er wieder nur die Daten, die vorher durch das Image definiert wurden.
Natürlich gibt es Anwendungsfälle, in denen die persistente Speicherung von Daten durch einen Container gewünscht ist, z.B. wenn
eine Datenbank in einem Container betrieben wird. Für diesen Fall gibt es das Konzept der \emph{Volumes}.
Ein Volume spiegelt einen definierten Speicherbereich aus dem Dateisystem des Containers auf das Host-System. 
Wird ein Container beendet und durch einen neuen ersetzt, bindet dieser das Volume in sein Dateisystem ein übernimmt dabei die Daten seines beendeten Vorgängers.

Die in diesem Abschnitt beschriebenen Konzepte sind ausreichend, um Container zu betreiben.
Nutzt man aber lediglich diese Konzepte, verhalten sich die Container im Prinzip wie leichtgewichtige virtuelle Maschinen.
Der eigentliche Vorteil von Containern, nämlich die horizontale Skalierbarkeit, wird damit noch nicht ausgenutzt.

Bie horizontale Skalierbarkeit ist abzugrenzen von der vertikalen Skalierbarkeit. Bei letzterer geht es darum, den Computer, der eine Anwendung betreibt,
um weitere Rechen- oder Speicherressourcen zu erweitern. Dieser Ansatz ist zwar leichter zu administrieren, da keine Änderungen an der Anwendung selbst
erforderlich sind, aber er stößt auch bald an seine physikalischen Grenzen. Horizontale Skalierung sieht im Gegensatz dazu vor, dass eine Anwendung
auf mehrere Rechner verteilt wird. Durch Hinzufügen weiterer Rechner können die zur Verfügung stehenden Ressourcen nahezu beliebig erweitert werden.
Zusätzlich ist auch eine geografische Optimierung möglich, indem zum Beispiel Server in der Nähe der erwarteten Clients platziert werden. 
Letztendlich trägt horizontale Skalierung auch zur Ausfallsicherheit einer Anwendung bei, da der Ausfall einzelner Instanzen durch andere Instanzen
aufgefangen werden kann.
Der Nachteil dieser Methode ist die deutliche erhöhte Komplexität, die mit dem Betrieb einer verteilten Anwendung einhergeht.

Im folgenden Abschnitt soll beschrieben werden, welche Werkzeuge Kubernetes anbietet, um mit dieser Komplexität umzugehen.

\section{Grundlegende Kuberneteskonzepte}
Ein durch Kubernetes verwaltetes System wird üblicherweise als \emph{Cluster} bezeichnet.
Ein Cluster besteht aus einer oder mehreren physischen und/oder virtuellen Maschinen, die darin eingebunden sind.
Jede Maschine wird dabei als ein \emph{Node} (deutsch: Knoten) bezeichnet. Diese Nodes werden wiederrum unterschieden in 
\emph{Master Nodes} und \emph{Worker Nodes}.
Für die meisten Cluster reicht ein Master Node aus. In Hochverfügbarkeitsszenarien oder bei mehreren
tausend Worker Nodes kann es jedoch auch mehrere Master Nodes geben \cite{Schmeling_Dargatz_2022}.
Wie die Namen vermuten lassen, ist der Master Node dafür verantwortlich für die Orchestrierung
des Systems verantwortlich während die Worker Nodes bestimmte Aufgaben ausführen \cite{Bentaleb_Belloum_Sebaa_El-Maouhab_2021}.
Für den Spezialfall eines \emph{Single-Node-Clusters} kann dieser einzelne Node auch die Rolle von sowohl
Master als auch Worker übernehmen.
Master und Worker Nodes bestehen zur Erfüllung ihrer Aufgaben aus unterschiedlichen Komponenten,
die im folgenden beschrieben werden.

\subsection{Master Node Komponenten}
\subsubsection{Etcd}
Bei \emph{Etcd} handelt es sich um einen \emph{key-value-store} (Schlüssel-Wert-Speicher),
der alle Informationen über den Zustand eines Kubernetes Clusters speichert.
Jede Änderung an den Cluster-Ressourcen wird im Etcd gespeichert und von dort wieder abgerufen.
Etcd wird auf jedem Master Node ausgeführt, wobei einer dieser Nodes als ``Anführer'' 
ausgewählt wird und dieser die Schreibanfragen verarbeitet.

\subsubsection{Kube-apiserver}
Die übliche Vorgehensweise um mit einem Kubernetes Cluster zu interagieren, ist das Kommandozeilenwerkzeug \emph{kubectl}.
Dieses Werkzeug ermöglicht sowohl die Abfrage als auch die Änderung von im Cluster vorhandenen Ressourcen.

Im Hintergrund stellt \emph{kubectl} eine Anfrage an den Kube-apiserver, der Endpunkte anbietet, 
um mit dem Cluster zu interagieren. Der Kube-apiserver ist somit dafür verantwortlich, die angefragen Informationen,
aus verschiedenen Quellen zu sammeln und gegebenenfalls andere Komponenten anzuweisen, bestimmte Änderungen vorzunehmen.

\subsubsection{Kube-Controller-Manager}
Neben dem tatsächlichen Zustand des Clusters enthält Etcd auch den gewünschten Zustand.
Sollten diese voneinander abweichen, ist es die Aufgabe des Kube-Controller-Managers,
dies zu erkennen und über eine Anfrage an den Kube-apiserver den gewünschten Zustand wiederherzustellen.


\subsubsection{Kube-Scheduler}
\label{sec:Kube-Scheduler}
Der Kube-Scheduler entscheidet, welche Container auf welchem Node betrieben werden.
Er verwaltet dazu Informationen über verfügbare Rechen- und Speicherressourcen sowie weiterer Eigenschaften
der Nodes. 
\subsection{Worker Node Komponenten}

Auf jedem Worker Node finden sich drei wesentliche Komponenten.
Die erste Komponente ist der \emph{Kubelet}. Das Kubelet kann als Schnittstelle
zu den Master Nodes gesehen werden. Sobald der Kube-Scheduler entschieden hat, auf welchem
Worker Node ein Container betrieben werden soll, bekommt das Kubelet auf diesem Node 
vom Kube-apiserver den Auftrag einen Container zu starten.

Kubelet gibt diesen Auftrag weiter an die Komponente \emph{Container Runtime}.
Diese übernimmt das eigentliche Starten des Containers, in dem sie das zugehörige Image
abruft (z.B. von DockerHub) und daraus einen Container erstellt.

Der physische ``Standort'' eines Containers, also der Node, auf dem er betrieben wird, ist insofern
flüchtig, als dass der Container jederzeit auf einem anderen Node neu ausgebracht werden könnte.
Um trotzdem einen Überblick zu behalten, welcher Container wo verortet ist, gibt es die Komponente
\emph{Kube-Proxy}. Auf jedem Node läuft genau eine Instanz des Kube-Proxy.


\subsection{Pods}
\emph{Pods} sind die kleinste adressierbare Einheit in einem Kubernetes Cluster. 
Sie sind die Entitäten, innerhalb derer der Anwendungscode ausgeführt wird.
Im Abschnitt \ref{sec:Kube-Scheduler} wurde beschrieben, dass Container auf Nodes betrieben werden.
Diese Darstellung ist allerdings nur indirekt richtig. Tatsächlich werden nicht Container sondern Pods auf Nodes betrieben.
Ein Pod wiederum kann einen oder mehrere Container enthalten. Enthält ein Pod nur einen Container,
kann dieser als einfacher ``Wrapper'' für diesen verstanden werden. 
Wenn ein Pod mehrere Container enthält, heißt das, dass diese Container nicht unabhängig von einander skalierbar sind.
Das bietet sich in der Regel dann an, wenn die Funktionen dieser Container eng gekoppelt sind und eine Instanz eines
Containers nur dann sinnvoll zu betreiben ist, wenn auch ihr ``Partner-''Container vorhanden ist.
Ein gängiges Beispiel ist der Betrieb eines Webservers in einem Container, dessen wesentliche Aufgabe es ist,
Anfragen an den eigentlichen Anwendungscontainer im gleichen Pod weiterzuleiten. 
\subsection{ReplicaSets}
Die gerade beschriebenen Pods können über \emph{kubectl} gestartet, gelöscht oder verändert werden.
Soweit betrachtet das aber nur einzelne Pods. Das Ziel von Kubernetes ist es jedoch, 
Pods zu replizieren und gegebenenfalls auf mehreren Nodes gleichzeitig zu betreiben.
Dazu stehen \emph{ReplicaSets} zur Verfügung. 
In einem ReplicaSet ist definiert, welche Art von Pods (basierend auf einem gemeinsamen Image)
in welcher Konfiguration betrieben werden sollen. Zusätzlich wird hier festgelegt, wie viele \emph{Replikationen}
dieses Pods gleichzeitig vorhanden sein sollen.
Kubernetes versucht nun die geforderte Anzahl an Pods zu starten und aufrechtzuerhalten. 
Wird ein Pod des ReplicaSets gelöscht, wird automatisch einer neuer Pod gestartet, um diesen zu ersetzen.
\subsection{Deployments}
Ein \emph{Deployment} ist ein Konzept, das im Schwerpunkt genutzt wird, um Änderungen an Anwendungen auszubringen.
Das Szenario, das ein Deployment notwendig macht, kann wie folgt beschrieben werden:

Eine Anzahl von Pods wurde im Rahmen eines ReplicaSets auf Basis eines gemeinsamen Images erstellt.
Es stellt sich heraus, dass das Image einen Fehler hat oder aus anderen Gründen aktualisiert werden muss.
Das ReplicaSet kann so konfiguriert werden, dass es zukünftig das neue Image für seine Pods verwendet.
Da die Pods aber bereits laufen, wird an diesen keine Änderung mehr vorgenommen. Es ist zwar möglich,
einzelne Pods zu löschen, die dann automatisch durch neue Pods, basierend auf dem neuen Image, ersetzt würden.
Gerade mit steigender Anzahl an Pods ist dieser Prozess allerdings höchst 
fehleranfällig, da es schwer zu überblicken ist, welche Pods noch mit dem alten und welche schon mit dem
neuen Image laufen. Alternativ könnten auch das gesamte ReplicaSet und damit alle zugehörigen Pods gelöscht werden.
Das kann in einer Testumgebung problemlos möglich sein. In einer Produktivumgebung möchte man derartige
Ausfälle aber in der Regel vermeiden.

Um diese Probleme zu bewältigen, kann ein Deployment erstellt werden.
Ein Deployment enthält wiederum ein ReplicaSet. Tritt nun das oben aufgeführte Szenario ein und ein Image
muss aktualisiert werden, reicht es, das Deployment dafür anzupassen.
Kubernetes stellt dann sicher, dass das alte ReplicaSet und die dazugehörigen Pods gelöscht werden.
Damit es dadurch aber keine Serviceunterbrechungen gibt, wird zunächst ein anderes ReplicaSet mit dem neuen
Image initialisiert. Während das alte ReplicaSet also schrittweise abgebaut wird, wird das neue ReplicaSet
schrittweise aufgebaut. Ein Deployment kann so konfiguriert werden, dass immer eine bestimmte Mindestanzahl an
Pods während der Aktualisierung zur Verfügung steht.
Ein weiterer Vorteil von Deployments ist, dass sie anhand eines einzelnen Befehls auf die vorherige Version
zurückgesetzt werden können (sog. \emph{Rollback}). Das kann hilfreich sein, wenn sich nach Ausbringung eines
Updates herausstellt, dass dieses doch nicht den Erwartungen entspricht und man stattdessen bei der vorherigen
Version bleiben möchte.

\subsection{Services}
Innerhalb eines Kubernetes Clusters besteht ein internes Netzwerk. Dementsprechend hat auch jeder Pod
eine eigene IP-Adresse, über die er erreichbar ist. Diese IP-Adresse wird automatisch zugewiesen wenn der Pod startet.
Auf diese Weise können Anwendungen innerhalb des Clusters über Pod-Grenzen hinweg miteinander kommunizieren.
Die Herausforderung in diesem Kontext ergibt sich daraus, dass Pods jederzeit beliebig gelöscht
und wieder neu gestart werden können. Da dabei auch jedes Mal eine neue IP-Adresse vergeben wird,
ist die direkte Pod-Adressierung wenig zuverlässig. Außerdem muss davon ausgegangen werden, dass
ein in einem ReplicaSet mehrere Replikationen eines Pods vom gleichen Typ vorliegt, wodurch unklar wird
an welcher der gleichartigen Pods eine Anfrage gesendet werden sollte. Im schlimmsten Fall gehen alle Anfragen
an denselben Pod des ReplicaSets, wodurch dieser überlastet wäre, während alle anderen Pods im Leerlauf sind.

Diesem Umstand wird mit \emph{Services} begegnet. Ein Service hat eine eigene IP-Adresse und ihm ist eine Menge
von Pods zugeordnet. Anfragen können nun an den Service gestellt werden und dieser entscheidet selbstständig,
an welchen seiner zugehörigen Pods diese Anfrage weitergegeben wird. Der Service übernimmt damit also 
nicht nur die Aufgabe der einfachen Adressierbarkeit sondern auch die der Lastverteilung (sog. \emph{Load Balancing}).

Natürlich kann auch ein Service gelöscht und neugestartet werden, wodurch ihm auch eine neue IP-Adresse zugewiesen wird.
Es ist also in der Regel nicht zielführend einen Service anhand seiner IP-Adresse anzusprechen.
Damit das nicht notwendig ist, verwaltet Kubernetes ein internes Domain-Name-System (DNS), in dem
der Name des Services automatisch registriert wird.

Kubernetes unterscheidet zwischen den Service-Typen \emph{ClusterIP} und \emph{NodePort}.
Während ClusterIP genutzt wird, um einen Services (und die zugehörigen Pods) \emph{innerhalb} eines Clusters
einheitlich adressierbar zu machen, wird NodePort verwendet, um einen Service von \emph{außerhalb}
über die öffentlichen IP-Adressen der Nodes des Clusters an einem zufällig gewählten Port erreichbar zu machen.

\subsection{Ingress}
Die Verwendung eines NodePort-Services wird meist nur zu Testzwecken verwendet. In einer Produktivumgebung
ist die gängigere Lösung, um das Cluster von außerhalb erreichbar zu machen, die Verwendung eines \emph{Ingress}.
In der Definition eines Ingress kann ein Hostname und ein dazugehöriger interner Service gesetzt werden.
Anfragen an diesen Hostname werden dann an den zugehörigen Service weitergeleitet. Über unterschiedliche Hostnames
oder Subdomains, die mit verschiedenen Services verknüpft werden, können Anfragen an den Ingress flexibel
an die richtigen Services weitergeleitet werden. Das Verhalten eines Ingress kann mit dem eines \emph{Reverse Proxy}
verglichen werden.

Kubernetes stellt dabei natürlich nicht sicher, dass die entsprechenden Hostnames in öffentlichen DNS Services auch mit
der öffentlichen IP-Adresse des Ingress verbunden sind.

\subsection{Volumes}
Das Konzept der \emph{Volumes} und deren Notwendigkeit wurde bereits in Kapitel \ref{sec:Container} eingeführt.
Kubernetes unterscheidet in diesem Zusammenhang zwischen \emph{PersistentVolumes} (PV) und \emph{PersistentVolumesClaims} (PVC).
Bei PV handelt es sich um eine Abstraktion eines Speicherbereichs des Hostsystems.
Ein PVC beansprucht ein derartiges PersistentVolume und spiegelt es in einen
definierten Speicherbereich zugehöriger Pods. Um ein PVC zu nutzen wird dieses innerhalb eines Deployments festgelegt.
Alle zu diesem Deployment gehörigen Pods teilen sich also dasselbe Volume.


\subsection{ConfigMaps und Secrets}
Die Daten, die von einer Anwendungen in Volumes gespeichert werden, unterliegen oft regelmäßigen Veränderungen
(z.B. bei einer Datenbank). Darüber hinaus benötigen Anwendungen aber auch einige Informationen, die sich nur 
selten verändern. Dabei handelt es sich häufig um Konfigurationsdaten, die zum Beispiel die URL eines anderen
Services enthalten oder Informationen über die Umgebung, in der die Anwendung betrieben wird.
Derartige Daten können Pods über \emph{ConfigMaps} in Form von \emph{Key-Value-Paaren} bereitgestellt werden.

Neben ConfigMaps gibt es auch \emph{Secrets}, die ihre Daten ebenfalls in Key-Value-Paaren speichern.
Der Unterschied ist, dass die Werte, die in Secrets eingetragen werden, üblicherweise geheim bleiben sollen und
dass sie base64 kodiert sind. Innerhalb der Pods werden die Werte automatisch dekodiert.
Es ist allerdings zu beachten, dass das Eintragen von Geheimnissen in Secrets sie nicht automatisch sicher
aufbewahrt. Davon ab, dass natürlich niemand Unbefugtes Zugriff auf die Konfigurationsdateien der Secrets haben sollte,
sind auch noch weitere Maßnahmen für einen sicheren Betrieb erforderlich, da die Secrets ansonsten
unverschlüsselt im Etcd gespeichert werden.

Die Daten aus ConfigMaps und Secrets werden den Anwendungen innerhalb der Pods als Umgebungsvariablen
bereitgestellt und können darüber ausgelesen werden.

Eine ausführlichere Erläuterung und Beispiele der in diesem Kapitel aufgeführten Komponenten 
kann bei Schmeling und Dargatz \cite{Schmeling_Dargatz_2022} gefunden werden.

% \subsection{Namespaces?}

% Einführung in praktisches Beispiel, aber 
% Erst Motivation für Simulationen 
	% Vorstellung der realen Problemstellung (Hohe Kosten für Betrieb (s. auch 10408594 S. 1), Vorhersage von effizientem Ressourceneinsatz)
% Voraussetzung für Simulation (System- und Datenanalyse)
% Herausforderungen bei Simulationen (Hohe Anzahl an möglichen Parameterkombinationen)

% Dann konkretes Beispiel

\section{Flottensimulation}
Für ein besseres Verständnis der grundlegedenden Konzepte von Kubernetes sollen diese anhand eines
Anwendungsfalls vorgestellt werden.
Vor der Implementierung des Anwendungsfalls muss aber die Motviation für diesen geklärt werden.

Konkret soll der Anwendungsfall die simulationsgestützte Analyse einer Fahrzeugflotte behandeln.
Im Kern handelt es hier um eine Abwandlung eines Warteschlangenmodells \cite{Leonelli2021}, bei dem
``Kunden'' (hier: Fahrzeuge) an einem ``Dienstleistungspunkt'' (hier: Instandsetzungseinrichtung) ankommen 
und dort bedient (hier: instandgesetzt) werden.
Wenn die Kapazitäten des Dienstleistungspunkts voll ausgelastet sind und trotzdem neue Kunden ankommen, entsteht eine Warteschlange.

Der grundlegende Ablauf in dem verwendeten Modell ist in Abbildung \ref{fig:Simulationsmodell} dargestellt.
Die Simulation enthält eine Menge an Fahrzeugen, die einer nicht näher definierten Nutzung unterliegen. 
Für jeden Iterationsschritt besteht eine Wahrscheinlichkeit von $\lambda$, dass das Fahrzeug ausfällt.
Tritt dieser Fall ein, wird das Fahrzeug einer Instandsetzungseinrichtung zugewiesen. 
Eine Instandsetzungseinrichtung hat einen oder mehrere Arbeitsplätze und diese enthalten wiederum Arbeiter.
Hat die Instandsetzungseinrichtung einen freien Arbeitsplatz, wird das defekte Fahrzeug diesem zugewiesen und
dort mit einer Rate instandgesetzt, die abhängig von der Anzahl und Fertigkeit der zugehörigen Arbeiter ist.
Falls gerade keine freien Kapazitäten vorhanden sind, verbleibt das Fahrzeug in der Warteschlange der jeweiligen Einrichtung.
Ist die Instandsetzung abgeschlossen, verlässt das Fahrzeug die Instandsetzungseinrichtung und wird weiter genutzt.

\begin{figure}
	\label{fig:Simulationsmodell}
	\includegraphics[width=\textwidth]{media/Simulationsmodell.png}
\end{figure}

Bereits in diesem vergleichsweise simplen Modell sind mehrere verschiedene Eingangsparameter (im Folgenden als \emph{Faktoren} bezeichnet)
und Ausgangsvariablen (im Folgenden als \emph{Key-Performance-Indicators bzw. KPI} bezeichnet) denkbar.

Beispiele für Faktoren:
\begin{enumerate}
	\item Anzahl der Fahrzeuge
	\item Ausfallwahrscheinlichkeit der Fahrzeuge $\lambda$
	\item Höhe der Instandsetzungskapazitäten
	\item Reparaturrate der Fahrzeuge
\end{enumerate}
Beispiele für KPI:
\begin{enumerate}
	\item Anzahl einsatzbereiter Fahrzeuge
	\item Dauer von Reparaturen
	\item Länge der Warteschlangen
	\item Höhe der Wartezeiten
	\item Auslastung der Instandsetzungseinrichtungen
\end{enumerate}

Ziel eines solchen Modells ist die Optimierung eines realen Systems mithilfe der Simulation.
Dazu wird das reale System mit möglichst allen relevanten Eigenschaften in der Simulation nachgebildet.
Dabei werden auch alle Faktoren bestimmt, von denen man im realen System erwartet, dass sie einen Einfluss auf die KPI
haben könnten und die auch aktiv verändert werden können.
Um eine möglichst optimale Konfiguration der Faktoren zu identifizieren, wird die Simulation 
mit vielen verschiedenen Konfigurationen durchgeführt und die dazugehörigen KPI ausgewertet.
Diese Methode ist auch unter dem Begriff \emph{Data Farming} bekannt. % Quelle?

In einer Simulation werden statt fester Werte oft Verteilungsfunktionen verwendet, aus denen Einflussgrößen abgeleitet werden. 
Aufgrund dieser Eigenschaft wird ein Simulationsmodell in der Regel mehrfach
mit gleichen Faktoren ausgeführt, um die stochastischen Effekte in den KPI zu beobachten und
um einzelne Ausreißer auszuschließen. 

Eine populäre Klassifizierung von Analysemodellen erfolgte durch Gartner \cite{Gartner} (vgl. Abbildung \ref{fig:gartner-modell}).
Diese Klassifizierung geht von vier Stufen von Analysemethoden aus, die anhand der Skalen \emph{Schwierigkeit} und \emph{Mehrwert} sortiert werden.
Beginnend bei niedriger Schwierigkeit und bei niedrigem Mehrwert bis hin zu hoher Schwierigkeit und hohem Mehrwert
handelt es sich dabei um:
\begin{enumerate}
	\item Deskriptive Analyse
	\item Diagnostische Analyse
	\item Prediktive Analyse
	\item Präskriptive Analyse
\end{enumerate}

\begin{figure}
	\label{fig:gartner-modell}
	\includegraphics[width=\textwidth]{media/analytic-maturity.png}
\end{figure}

Ein Simulationsmodell der hier vorliegenden Art fällt grundsätzlich in den Bereich der präskriptiven Analyse, 
da es nicht nur die Vorhersage zukünftiger Systemzustände erlaubt, sondern auch Faktoren identifiziert,
die verändert werden können, um einen bestimmten Zustand zu erreichen.

Ein Beispiel für eventuelle Nutzer einer Simulation wie der hier genutzten, könnten 
Betreiber größerer Fahrzeugflotten (z.B. LKW oder Busse) sein. 
Fahrzeuge, die sich lange in Instandsetzung befinden, können in dieser Zeit keinen Mehrwert generieren.
Es besteht also ein Interesse daran, immer ausreichende Instandsetzungskapazitäten vorhalten zu können.
Da der Unterhalt derartiger Kapazitäten aber mit hohen Kosten verbunden ist, ist es von großem
Interesse, die \emph{optimale} Instandsetzungskapazität für eine gegebene Flotte zu identifizieren.

Der Bedarf an skalierbarer Infrastruktur zum Betrieb der Simulation ergibt sich daher wie folgt:
Bereits bei den vier oben aufgeführten Faktoren, mit denen das vorliegende Simulationsmodell betrieben werden könnte,
ergibt sich eine große Anzahl an Faktorkombinationen.
Angenommen jeder Faktor soll um fünf Stufen variiert werden. Dann ergeben sich daraus
$5^4 = 625$ Kombinationen. Aufgrund stochastischer Effekte empfiehlt es sich, jede einzelne
Kombination mehrfach zu wiederholen. Davon ausgehend, dass jede Kombination 20 mal wiederholt wird,
ergeben sich daraus schon $625 * 20 = 12.500$ Simulationsläufe. Bei einer angenommenen Laufzeit von ca. 10 Sekunden
pro Simulationslauf und rein serieller Ausführung ergibt sich daraus bereits eine Gesamtlaufzeit von
knapp 35 Stunden.

Bei Modellen für realistische Szenarien ist natürlich von mehr Faktoren und auch von einer höheren Zeit 
für einzelne Simulationsläufe auszugehen. Um einen vollständigen Simulationslauf möglichst schnell abzuschließen
und folglich auswertbare Ergebnisse zu erhalten, bietet es sich an, die Rechenlast zu verteilen.  





\begin{figure}
	\includegraphics[width=\textwidth]{media/Microservice-Architektur.png}
\end{figure}




\subsection{Schritte zum Aufsetzen und zum Betrieb der Flottensimulation}

\section{Limitationen}
% Begrenzte Hardware und hohe Kosten bei Cloud Providern 

\section{Fazit}

\section{Ausblick}
% Nutzung einer anderen Cluster Engine als minikube (s. auch 10408202 für Rancher Alternative)
% Erweiterung des Simulationsmodells um Condition based maintenance/digital twin (s. auch 10408594)
% Erhöhung der Komplexität des Modells durch weitere Input-Faktoren
% FastAPI um Authentisierung ergänzen
\section{Anhang}


% \bibliographystyle{babplain-fl}
% \bibliography{literature}

\printbibliography

\end{document}